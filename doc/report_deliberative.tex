\documentclass[11pt]{article}

\usepackage{amsmath}
\usepackage{textcomp}
\usepackage{mathtools}

% Add other packages here %


% Put your group number and names in the author field %
\title{\bf Excercise 3\\ Implementing a deliberative Agent}
\author{Group 39 : Hugo Bonnome, Pedro Amorim}


% N.B.: The report should not be longer than 3 pages %


\begin{document}
\maketitle

\section{Model Description}

\subsection{Intermediate States}
% Describe the state representation %
First of all let us specify the elements that we need to define the state.
$ C $ is the set of all cities in the topology. $ D $ is the set of all the
delivery tasks currently in the topology including the ones that are being
delivered.
A task $ t $ is defined by $ t
\coloneqq (o, d) $ where $ o, d \in C $, with $ o $ being the city where the
task originated and $ d $ the destination city of the task.
The state $ s $ of an agent is defined by $$ s \coloneqq (c, A, T) $$ where $c
\in C$ is the city where the agent currently is, $ A \subseteq D $ is the set of
avalaible tasks in the topology, $ T \subseteq D $ is the set of tasks that the
agent is currently delivering.

\subsection{Goal State}
% Describe the goal state %
With the intermediates states defined we can define the goal state by $$ g
\coloneqq (c, \O, \O) $$ where $ c \in C $ can be any city of the topology and
the empty sets representing the fact that all tasks of the topology have been
delivered.

\subsection{Actions}
% Describe the possible actions/transitions in your model %
An agent can do the following actions:
\begin{itemize}
\item Move to a neighbour $ n $ of the current city $ c $. $$ (c, A, T)
\rightarrow (n, A, T) $$
\item Pick up a task $ t $ in the current city $ c $. $$ (c, A, T) \rightarrow
  (c, A', T') $$ where $ A' = A \setminus (t) $ and $ T' = T \cup (t) $
\item Deliver a task $ t $ in the current city $ c $. $$ (c, A, T) \rightarrow
  (c, A, T') $$ where $ T' = T \setminus (t) $
\end{itemize}

\section{Implementation}

\subsection{BFS}
% Details of the BFS implementation %

\subsection{A*}
% Details of the A* implementation %

\subsection{Heuristic Function}
% Details of the heuristic functions: main idea, optimality, admissibility %


\section{Results}

\subsection{Experiment 1: BFS and A* Comparison}
% Compare the two algorithms in terms of: optimality, efficiency, limitations %
% Report the number of tasks for which you can build a plan in less than one minute %

\subsubsection{Setting}
% Describe the settings of your experiment: topology, task configuration, etc. %

\subsubsection{Observations}
% Describe the experimental results and the conclusions you inferred from these results %


\subsection{Experiment 2: Multi-agent Experiments}
% Observations in multi-agent experiments %

\subsubsection{Setting}
% Describe the settings of your experiment: topology, task configuration, etc. %

\subsubsection{Observations}
% Describe the experimental results and the conclusions you inferred from these results %

\end{document}